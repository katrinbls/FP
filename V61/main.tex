\documentclass[titlepage=firstiscover, captions=tableheading, bibliography=totoc]{scrartcl}
\usepackage[autostyle=true,german=quotes]{csquotes}
\usepackage{scrhack}
\usepackage{caption}
\usepackage[aux]{rerunfilecheck}
\usepackage{subcaption}        
\usepackage{fontspec}
\usepackage[dvips]{graphicx}
\usepackage{floatflt,epsfig} 
    
\usepackage{polyglossia}
\setmainlanguage{german}

\usepackage[unicode]{hyperref}
\usepackage{bookmark}
\title{V27\\ Der Helium-Neon-Laser}
\author{
Miriam Simm\\
\texorpdfstring{\href{mailto:miriam.simm@tu-dortmund.de}{miriam.simm@tu-dortmund.de}\and}{,}
Katrin Bolsmann\\
\texorpdfstring{\href{mailto:katrin.bolsmann@tu-dortmund.de}{katrin.bolsmann@tu-dortmund.de}}{}
}
\date{Durchführung: 15.06.2020 \\ Abgabe: -.06.2020}
\usepackage{amsmath} 
\usepackage{amssymb} 
\usepackage{mathtools}
\usepackage[
    math-style=ISO,
    bold-style=ISO,
    sans-style=italic,
    nabla=upright,
    partial=upright,
]{unicode-math}
    
\setmathfont{Latin Modern Math}

\usepackage[
  locale=DE,
  separate-uncertainty=true, 
  per-mode=symbol-or-fraction,
]{siunitx}

\usepackage{multicol}
\setlength{\columnsep}{1pt} %space between columns 

\usepackage{booktabs}
\usepackage[x11names, table]{xcolor}
\usepackage{graphicx}
\usepackage{grffile}
\usepackage{xfrac}
\usepackage{xcolor}

\usepackage{float}
\floatplacement{figure}{h}
\floatplacement{table}{h}
\usepackage[
  section,
  below,
]{placeins}

\usepackage{expl3}
\usepackage{xparse}
\ExplSyntaxOn
\NewDocumentCommand \E {} {\symup{e}}
\ExplSyntaxOff

% Literaturverzeichnis
\usepackage[
  backend=biber,
]{biblatex}
% Quellendatenbank
\addbibresource{literatur.bib}

\usepackage[
  version=4,
  math-greek=default,
  text-greek=default,
]{mhchem}
 

\raggedcolumns

\begin{document}

\maketitle

\section{Auswertung}
\subsection{Überprüfung der Stabilitätsbedingung}
Die Messwerte zur Untersuchung der Stabilitätsbedingung sind in der Tabelle \ref{tab:atab1} des Anhangs zu finden.
Nach \eqref{eq:} zeigt die Leistung einen quadratischen Verlauf, somit werden die Messwerte mit an eine quadratische Funktion
\begin{equation}
    I(L) = a L^2 + b L + c \label{eq:fit}
\end{equation}
gefittet. 
Um die Messwerte besser vergleichen zu können werden diese normiert und die Nullleistung, für welche ein Wert von $\Phi_0 = (16.63 \pm 0.043)\si{\micro\W}$ gemessen wurde, abgezogen.
Die enstprechenden Fitparamter sind in Tabelle \ref{tab:atab2} zu finden, wobei zurerst die Messreihen separat und anschließend alle Messwerte zusammen betrachtet werden. 
Die Messdaten und die nach \eqref{eq:fit} errechneten Ausgleichspolynome sind in den Abbildungen \ref{fig:afig1} und \ref{fig:afig2} dargestellt.
\noindent
\FloatBarrier
\begin{figure}[h]
\centering
\includegraphics[width=\textwidth]{Stabilität.pdf}
\caption{Messwerte der Strahlungsleistung aller drei Messreihen in Abhängigkeit der Resonatorlänge und Ausgleichsrechnung.}
\label{fig:afig1}
\end{figure}
\FloatBarrier
\noindent
\FloatBarrier
\begin{figure}[h]
\centering
\includegraphics[width=\textwidth]{Stabilität_all.pdf}
\caption{Messwerte der Strahlungsleistung in Abhängigkeit der Resonatorlänge und Ausgleichsrechnung.}
\label{fig:afig2}
\end{figure}
\FloatBarrier
\noindent
\FloatBarrier
\begin{table}[h]
    \centering
    \caption{Parameter der Gleichung \eqref{eq:fit}.}
    \label{tab:atab2}
    \begin{tabular}{l c c c c}
        \toprule
        {} & {Messung 1} & {Messung 2} & {Messung 3} & {alle Messwerte} \\
        \midrule
        $a / \si{\micro\W\per\cm\squared}\cdot10^{3}$ & $(1,084 \pm 0,731)$ & $(0,190 \pm 0,816)$ & $(1,51 \pm 0,358)$ & $(0,451 \pm 0,370)$ \\
        $b / \si{\micro\W\per\cm}\cdot10^{2}$ & $(-19,951 \pm 11,712)$ & $(-6,460 \pm 14,048)$ & $(-26,332 \pm 5,742)$ & $(-9,607 \pm 6,066)$ \\
        $c / \si{\micro\W}$ & $(9,280 \pm 4,639)$ & $(4,547 \pm 5,998)$ & $(11,540 \pm 2,275)$ & $(5,156 \pm 2,464)$ \\
        \bottomrule
    \end{tabular}
\end{table}
\FloatBarrier

Diese und alle folgenen Ausgleichsrechnungen werden mit dem Paket $\texttt{SciPy.optimize.curve\_fit}$ durchgeführt, Fehlerrechnung erfolgt mittels dem $\texttt{Python}$-Paket $\texttt{uncertainties.unumpy}$.
Eine ausführliche Diskussion des Plots ist in Abschnitt \ref{sec: Diskussion} zu finden.

section{Untersuchung der TEM-Moden}




\section{Diskussion} \label{sec: Diskussion}

\appendix 
\section*{Anhang}\label{sec:Anhang}

\FloatBarrier
\begin{table}[h]
    \centering
    \caption{Messwerte der Strahlungsleistung $\Phi$ in Abhängigkeit der Resonatorlänge $L$.}
    \sisetup{table-format=2.1}
    \label{tab:atab1}
    \begin{tabular}{c c c c c c}
        \toprule
        \multicolumn{2}{c}{Messung 1} & \multicolumn{2}{c}{Messung 2} & \multicolumn{2}{c}{Messung 3} \\
        \cmidrule(lr){1-2}\cmidrule(lr){3-4}\cmidrule(lr){5-6}
        {$L/\si{\cm}$} & {$\Phi/\si{\micro\W}$} & {$L/\si{\cm}$} & {$\Phi/\si{\micro\W}$} & {$L/\si{\cm}$} & {$\Phi/\si{\micro\W}$} \\
        \midrule


        \bottomrule
    \end{tabular}
\end{table}
\FloatBarrier
\noindent


\nocite{wingate}
\nocite{*}
\printbibliography

\end{document}