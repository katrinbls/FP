\input{header.tex}

\begin{document}

\maketitle

\section{Auswertung}
\subsection{Überprüfung der Stabilitätsbedingung}
Die Messwerte zur Überprüfung der Stabilitätsbedingung sind in der Tabelle \ref{tab:atab1} des Anhangs zu finden.
Nach \eqref{eq:} zeigt die Leistung für zwei konkave Spiegel $r=1400\,\si{\mm}$ einen quadratischen Verlauf, somit werden die Messwerte an eine quadratische Funktion der Form
\begin{equation}
    \symup{\Phi}(L) = a L^2 + b L + c \label{eq:fit1}
\end{equation}
gefittet. 
Um die Messwerte besser vergleichen zu können werden diese normiert und die Nullleistung, für welche ein Wert von $\symup{\Phi} = (16.63 \pm 0.043)\si{\micro\W}$ gemessen wurde, abgezogen.
Die enstprechenden Fitparamter sind in Tabelle \ref{tab:atab2} zu finden, wobei zurerst die Messreihen separat und anschließend alle Messwerte zusammen betrachtet werden. 
Die Messdaten und die nach \eqref{eq:fit1} errechneten Ausgleichspolynome sind in den Abbildungen \ref{fig:afig1} und \ref{fig:afig2} dargestellt.
Eine ausführliche Diskussion der Plots ist in Abschnitt \ref{sec: Diskussion} zu finden.

Diese und alle folgenden Ausgleichsrechnungen werden mit dem Paket $\texttt{SciPy.optimize.curve\_fit}$ durchgeführt. Fehlerrechnung erfolgt mittels dem $\texttt{Python}$-Paket $\texttt{uncertainties.unumpy}$.
\FloatBarrier
\begin{table}[h]
    \centering
    \caption{Parameter der Gleichung \eqref{eq:fit1}.}
    \label{tab:atab2}
    \begin{tabular}{l c c c c}
        \toprule
        {} & {Messung 1} & {Messung 2} & {Messung 3} & {alle Messwerte} \\
        \midrule
        $a / \si{\micro\W\per\cm\squared}\cdot10^{3}$ & $(1,084 \pm 0,731)$ & $(0,190 \pm 0,816)$ & $(1,51 \pm 0,358)$ & $(0,451 \pm 0,370)$ \\
        $b / \si{\micro\W\per\cm}\cdot10^{2}$ & $(-19,951 \pm 11,712)$ & $(-6,460 \pm 14,048)$ & $(-26,332 \pm 5,742)$ & $(-9,607 \pm 6,066)$ \\
        $c / \si{\micro\W}$ & $(9,280 \pm 4,639)$ & $(4,547 \pm 5,998)$ & $(11,540 \pm 2,275)$ & $(5,156 \pm 2,464)$ \\
        \bottomrule
    \end{tabular}
\end{table}
\FloatBarrier
\noindent
\FloatBarrier
\begin{figure}[h]
\centering
\includegraphics[width=\textwidth]{Stabilität.pdf}
\caption{Messwerte der Strahlungsleistung aller drei Messreihen in Abhängigkeit der Resonatorlänge und Ausgleichsrechnung für $r_1=r_2=1400\,\si{\mm}$.}
\label{fig:afig1}
\end{figure}
\FloatBarrier
\noindent
\FloatBarrier
\begin{figure}[h]
\centering
\includegraphics[width=\textwidth]{Stabilität_all.pdf}
\caption{Messwerte der Strahlungsleistung in Abhängigkeit der Resonatorlänge und Ausgleichsrechnung $r_1=r_2=1400\,\si{\mm}$..}
\label{fig:afig2}
\end{figure}
\FloatBarrier
\noindent




\subsection{Untersuchung der TEM-Moden}
Die Messdaten zur Untersuchung der $\text{TEM}_{0,0}$-Mode sind in Tabelle \ref{tab:atab3} des Anhangs zu finden. 
Die Grundmode wird an eine Gaußfunktion der Form
\begin{equation}
\symup{\Phi}(x) = \symup{\Phi}_0 \text{exp}\left(-2\left(\frac{(x-x_0)}{\omega}\right)^2\right) \label{eq:fit2}
\end{equation}
gefittet. 
Dabei stellt $\symup{\Phi}_0$ die maximale Strahlungsleistung und $x_0$ die transversale Verschiebung des Maximums dar. 
$\omega$ ist ein Maß für die Breite der Verteilung. 
Für diese drei freien Parameter ergeben sich die Werte
\begin{align}
\symup{\Phi}_0 &= (22,524 \pm 0,853)\,\si{\micro\watt} \\
x_0 &= (12,200 \pm 1,226)\,\si{\mm}\\
\omega &= (25,566\pm5,936)\,\si{\mm} \quad .
\end{align}
Die daraus resultierende Ausgleichsfunktion ist zusammen mit den Messwerten in Abbildung \ref{fig:afig3} aufgetragen.
\noindent
\FloatBarrier
\begin{figure}[h]
\centering
\includegraphics[width=\textwidth]{TEM00.pdf}
\caption{Messdaten der Strahlungsleistungs der $\text{TEM}_{0,0}$-Mode entlang der transversalen Achse und gefittete Gaußfunktion \eqref{eq:fit2}.}
\label{fig:afig3}
\end{figure}
\FloatBarrier

\subsection{Untersuchung der Polarisation}
Die für die Polarisationsuntersuchung aufgenommenen Messwerte sind in Tabelle \ref{tab:atab4} des Anhangs zu finden. 
In Abbildung \ref{fig:afig4} ist die Strahlungsleistung gegen den Winkel $\alpha$ des Polarisationsprismas aufgetragen.
\noindent
\FloatBarrier
\begin{figure}[h]
\centering
\includegraphics[width=\textwidth]{Polarisation.pdf}
\caption{Sinusförmiger Verlauf der Strahlungsleistung in Abhängigkeit der Winkels des Polarisationsprismas.}
\label{fig:afig4}
\end{figure}
\FloatBarrier
Die Ausgleichsfunktion hat die Form
\begin{equation}
\symup{\Phi}(\alpha) = \symup{\Phi}_0 \sin^2{\alpha-\alpha_0} \quad ,
\end{equation}
wobei sich für die freien Parameter die Werte
\begin{align}
\symup{\Phi}_0 & = (790,013\pm 5,936)\, \si{\micro\watt}\\
\alpha &= (-0,353\pm 0,007)\, \mathrm{rad}
\end{align}
ergeben. 
Die Minimalleistung ist demnach bei einem Winkel von
\begin{align}
\alpha - \alpha_0 &= 2\pi n \qquad \qquad n \in \mathbb{N} \\
\Leftrightarrow \qquad \quad \alpha &=  2\pi n + \alpha_0 \\
&\stackrel{n = 0}{=} (-0,353\pm 0,007)\, \mathrm{rad} \\
&= -20.2° \pm 0.4°
\end{align}
erreicht. Somit folgt, dass das Licht des HeNe-Laser linear in einer um $-20,2°$ gekippten Ebenen polarisiert ist.

\subsection{Berechnung der Wellenlänge}
Die Wellenlänge wird anhand des Interferenzmusters, welches sich durch Beugung an einem Gitter ergibt, errechnet.
Hierzu wurden drei Gitter mit unterschiedlichen Liniendichten betrachtet.
Die Messwerte sind in Tabelle \ref{tab:atab5} des Anhangs zu finden. 
Wobei $n$ jeweils die Ordnung des Hauptmaximums beschreibt, also um das wievielte Maximum ausgehend vom zentralen Maximum es sich handelt.

Die Wellenlänge wird mittels der Formel \eqref{eq:Wellenlänge} berechnet.
Somit ergibt sich für jedes vermessene Maximum ein Wert für die Wellenlänge, welche in Tabelle \ref{tab:atab6} angegeben sind.

\FloatBarrier
\begin{table}[h]
    \centering
    \caption{Nach \eqref{eq:Wellenlänge} berechneten Werte für die Wellenlänge, aller drei Gitter.}
    \sisetup{table-format=2.1}
    \label{tab:atab6}
    \begin{tabular}{c c c c c c}
        \toprule
        \multicolumn{1}{c}{100 Linien/\si{\mm}} & \multicolumn{1}{c}{600 Linien/\si{\mm}} & \multicolumn{1}{c}{1200 Linien/\si{\mm}} \\
        \cmidrule(lr){1-1}\cmidrule(lr){2-2}\cmidrule(lr){3-3}
        {$\lambda /\si{\nm}$} & {$\lambda /\si{\nm}$} & {$\lambda /\si{\nm}$} \\
        \midrule
        643,232\pm 1,180 & 620,884 \pm 1,270 & 604,1 \pm 0,7 \\
        630,291\pm 1,142 & 624,788 \pm 0,650 & \\
        627,031\pm 1,114 & 633,478 \pm 1,287 & \\
        624,156\pm 1,078 & & \\
        626,156\pm 1,040 & & \\
        628,286\pm 0,993 & & \\
        628,460\pm 0,933 & & \\
        628,725\pm 0,865 & & \\
        630,263\pm 0,787 & & \\
        661,531\pm 1,213 & & \\
        630,291\pm 1,142 & & \\
        627,031\pm 1,114 & & \\
        628,333\pm 1,084 & & \\
        \\
        $\text{Mittelwerte}\,\bar{\lambda}/\si{\mm}$ & &\\
        \cmidrule(lr){1-1}\cmidrule(lr){2-2}\cmidrule(lr){3-3}
        631,8 \pm 1,1 & 626,4 \pm 1,1 & 604,1 \pm 0,7 \\
        \bottomrule
    \end{tabular}
\end{table}
\FloatBarrier
\noindent
Werden die Mittelwerte für die verschiedenen Gitter erneut gemittel ergibt sich für die Wellenlänge ein Wert von
\begin{equation}
\lambda = (620,8 \pm 0,5) \, \si{\nm} \qquad .
\end{equation}


\section{Diskussion} \label{sec: Diskussion}
\subsection*{Überprüfung der Stabilitätsbedingung}
In diesem Versuch wurden für den Resonantor zwei konkav gekrümmte Spiegel mit dem Radius $r=1400\,\si{\mm}$ verwendet.
Der erwartete Intensitätsverlauf ist demnach quadratisch im Abstand der Spiegel und besitzt ein Minimum bei $L = r$, in welchem die Intensität verschwindet.

In Abbildung \ref{fig:afig1} zeigen die Messwerte der 2. Messung entgegen der Erwartungen einen beinahe linearen Verlauf im beobachteten Bereich. 
Zwar ist bei den anderen beiden Messungen ein quadratischer Verlauf zu erkennen, jedoch weist keine der errechneten Ausgleichsfunktionen ein Minimum bei $L = 1,4\,\si{\m}$ auf.
Die Minima der 1. und 3. Messreihe liegen in einem Bereich zwischen $(0,85-0,95)\,\si{\m}$.

Die Ausgleichfunktion in Abbildung \ref{fig:afig2}, welche sich aus den Messwerten aller drei Messreihen ergibt, zeigt einen etwas besseren Verlauf.
Es ist ein Minimum bei circa $L=1,05\,\si{\m}$ zu erkennen, welches ein wenig näher an der Lage des erwarteten Minimum bei $L=1,40\,\si{\m}$ liegt.

Wahrscheinlich haben Fehler und Ungenauigkeiten bei dem Messprozessen zu diesen Abweichungen geführt.
Somit musste der HeNe-Laser bei Veränderung des Abstandes neu justiert werden und je nachdem wie gut der Laser justiert ist, schwankt die Strahlungsleistung stark.
Im Optimalfall wird immer die maximale Strahlungsleistung für die entsprechende Resonatorlänge gemessen, diese ist allerdings schwer zu erreichen.

Eine weitere Möglichkeit die Ergebnisse zu verbessern, wäre es mehr Messwerte vor allem für große Resonatorlängen zu nehmen.
Allerdings stellte sich die Messung für große Resonatorlängen, ab etwa $1\,\si{m}$ als sehr schwierig dar, da aufgrund der geringen Leistung die Justage des Lasers kaum möglich war.


\subsection*{Untersuchung der TEM-Moden}
Die Grundmode konnte bei der Durchführung des Versuches auf dem Schirm sichtbar gemacht und somit vermessen werden. 
Jedoch ist in Abbildung \ref{fig:afig3} deutlich zu sehen, dass mehr Werte gerade in dem Bereich geringer Intensität hätten genommen werden müssen.

Ursprünglich ist auch die Vermessung der ersten angeregten Mode TEM$_{\text{01}}$ Teil dieses Versuches.
Allerdings ist es auch nach etlichen Versuchen nicht gelungen diese auf dem Schirm sichbar zu machen.
Grund hier für könnte die Qualität des verwendetem Haares sein. 
Möglicherweise ist dieses bereits ein wenig ausgefranst oder anders beschädigt, sodass es sich nicht zur Unterdrückung von Moden eignet.

\subsection*{Untersuchung der Polarisation}
Die Messwerte zur Polarisationsuntersuchung zeigen genau den zu erwarteten $sin^2$-Verlauf.
Dementsprechend repräsentiert die Ausgleichsfunktion diese Werte sehr gut und es kann angenommen werden, dass die Polarisationsrichtung ziemlich genau dem berechneten Wert von $\alpha = -20.2° \pm 0.4°$ entspricht.

\subsection*{Berechnung der Wellenlänge}
Die berechneten Wellenlängen sind noch einmal in Tabelle \ref{tab:atab7} zusammengefasst und die Abweichung zu der tatsächlichen Wellenlänge eines Helium-Neon Lasers $\lambda = 632,8\,\si{\nm}$ angegeben.
\FloatBarrier
\begin{table}[h]
    \centering
    \caption{}
    \sisetup{table-format=2.1}
    \label{tab:atab7}
    \begin{tabular}{l c c}
        \toprule
        {} & {$\lambda /\si{\nm}$} & {Abweichung}\\
        \midrule
        100 Linien/\si{\mm} & 631,8 \pm 1,1 & 0,16\% \\
        600 Linien/\si{\mm} & 626,4 \pm 1,1 & 1,01\%\\
        1200 Linien/\si{\mm} & 604,1 \pm 0,7& 4,54\%\\
        \\
        \midrule
        Mittelwert & 620,8\pm 0,5 & 1,90\%\\ 
        \bottomrule
    \end{tabular}
\end{table}
\FloatBarrier
\noindent
Die Messung mit dem Gitter bei welchem der Linienabstand $0,01\,\si{\mm}$ beträgt, liefert das beste Ergebnis mit nur $0,16\%$ Abweichung vom realen Wert. 
Dies liegt wahrscheinlich daran, dass der Abstand der Maxima für geringe Liniendichten abnimmt. 
Somit konnten für das Gitter mit 100 Linien/$\si{\mm}$ die meisten Messwerte aufgenommen werden, während bei dem Gitter mit 1200 Linien/$\si{\mm}$ lediglich ein Abstand gemessen werden konnte.
Trotzdem liefern auch die anderen beiden Messungen sehr gute Werte mit jeweils Abweichungen von unter $5\%$.

\newpage
\appendix 
\section*{Anhang}\label{sec:Anhang}

\FloatBarrier
\begin{table}[h]
    \centering
    \caption{Messwerte der Strahlungsleistung $\symup{\Phi}$ in Abhängigkeit der Resonatorlänge $L$.}
    \sisetup{table-format=2.1}
    \label{tab:atab1}
    \begin{tabular}{c c c c c c}
        \toprule
        \multicolumn{2}{c}{Messung 1} & \multicolumn{2}{c}{Messung 2} & \multicolumn{2}{c}{Messung 3} \\
        \cmidrule(lr){1-2}\cmidrule(lr){3-4}\cmidrule(lr){5-6}
        {$L/\si{\cm}$} & {$\symup{\Phi}/\si{\micro\W}$} & {$L/\si{\cm}$} & {$\symup{\Phi}/\si{\micro\W}$} & {$L/\si{\cm}$} & {$\symup{\Phi}/\si{\micro\W}$} \\
        \midrule
        %Hier bitte die Daten der Intensitäts-Abstands Messung einfügen aus den Intensität_Abstand txt dateien


        \bottomrule
    \end{tabular}
\end{table}
\FloatBarrier
\noindent

\FloatBarrier
\begin{table}[h]
    \centering
    \caption{Messdaten der Strahlungsleistungs der $\text{TEM}_{0,0}$-Mode entlang der transversalen Achse.}
    \sisetup{table-format=2.1}
    \label{tab:atab3}
    \begin{tabular}{c c}
        \toprule
        {$x/\si{\mm}$} & {$\symup{\Phi}/\si{\micro\W}$}\\
        \midrule
        % hier bitte die Messwerte der TEM Mode aus der datei TEM00_Mode.txt einfügen :)

        \bottomrule
    \end{tabular}
\end{table}
\FloatBarrier
\noindent

\FloatBarrier
\begin{table}[h]
    \centering
    \caption{Messwerte zur Untersuchung der Polarisation des HeNe-Lasers in Abhängigkeit des Winkels des Polarisationsprismas.}
    \sisetup{table-format=2.1}
    \label{tab:atab4}
    \begin{tabular}{c c}
        \toprule
        {$\alpha/\mathrm{rad}$} & {$\symup{\Phi}/\si{\micro\W}$}\\
        \midrule
        % hier bitte die Messwerte der Polarisationsmessung aus der datei Polarisation.txt einfügen :)

        \bottomrule
    \end{tabular}
\end{table}
\FloatBarrier
\noindent

\FloatBarrier
\begin{table}[h]
    \centering
    \caption{Gemessene Abstände der Haupmaxima zum zentralen Hauptmaximum. Es wird eine Ablesefehler von $0,1\,\si{\cm}$ angenommen. $L$ ist der Abstand zwischen Gitter und Schirm.}
    \sisetup{table-format=2.1}
    \label{tab:atab5}
    \begin{tabular}{c c c c c c}
        \toprule
        \multicolumn{2}{c}{100 Linien/\si{\cm}} & \multicolumn{2}{c}{600 Linien/\si{\cm}} & \multicolumn{2}{c}{1200 Linien/\si{\cm}} \\
        \multicolumn{2}{c}{L= 54,3\si{\cm}} & \multicolumn{2}{c}{L = 42,1\si{\cm}} & \multicolumn{2}{c}{L=42,1\si{\cm}} \\

        \cmidrule(lr){1-2}\cmidrule(lr){3-4}\cmidrule(lr){5-6}
        {$n$} & {$d / \si{\cm}$} & {$n$} & {$d / \si{\cm}$} & {$n$} & {$d / \si{\cm}$} \\
        \midrule
        %Hier bitte die Daten für die Wellenlängen bestimmung aus Gitter_xx.txt einfügen


        \bottomrule
    \end{tabular}
\end{table}
\FloatBarrier
\noindent

\nocite{wingate}
\nocite{*}
\printbibliography

\end{document}